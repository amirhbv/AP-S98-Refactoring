\documentclass{utap}

\usepackage{keyval}
\usepackage{calc}
\usepackage{multicol}
\usepackage{xepersian}
\usepackage{graphicx}
\graphicspath{ {./images/} }

\title{تمرین  بازسازی}
\author{
    \href{mailto:bardia.eghbali@gmail.com?subject=[AP\%20S98\%20Refactoring]\%20}{بردیا اقبالی},
    \href{mailto:ahhabibvand@gmail.com?subject=[AP\%20S98\%20Refactoring]\%20}{امیرحسین حبیب‌وند}
}
\course{برنامه‌سازی پیشرفته}
\deadline{جمعه ۱۶ فروردین ۱۳۹۸، ساعت ۲۳:۵۵}
\lecturer{رامتین خسروی}

\begin{document}

\lstset{
    numbers=left,
    frame=leftline,
}

\maketitle

\section{بازسازی}
تعاریف زیادی از
"ﮐﺪ ﺗﻤﯿﺰ"\LTRfootnote{Clean Code}
وجود دارد؛ اما احتمالا یکی از بهترین تعریف‌ها متعلق به
"ﺑﯿﺎرﻧﻪ اﺳﺘﺮاﺳﺘﺮوپ" \LTRfootnote{Bjarne Stroustrup}
خالق و توسعه‌دهنده‌ی زبان \lr{\texttt{C++}} است. وی ر تعریف خود از یک کد تمیز، دو مورد زیر را به عنوان معیار‌های اساسی یک کد تمیز بر می‌شمارد:

    \begin{itemize}
        \item 
منطق و الگوریتم کد باید آن‌قدر واضح و قابل‌فهم باشد که اشکالات و نقص‌های ﺟﺰﺋﯽ 

        \item
    \end{itemize}
 
	\begin{center}
	\end{center}

\end{document}
